\documentclass[12pt]{article}
\usepackage{amssymb,amsfonts,amsthm,amsmath}
%\usepackage{yfonts}
\usepackage{mathrsfs,pifont}
\usepackage{graphicx}

%\usepackage{mathtools}
%\usepackage{yfonts}
%\usepackage{mathrsfs,pifont}
%\usepackage{slashed,mathabx} 


\usepackage[all]{xy}

\usepackage{tikz}
\usetikzlibrary{matrix,arrows}
\usepackage{hyperref}
\usepackage[backrefs,msc-links]{amsrefs}

\newcommand{\C}{\mathbb{C}}
\newcommand{\bbA}{\mathbb{A}}
\newcommand{\Z}{\mathbb{Z}}
\newcommand{\R}{\mathbb{R}}
\newcommand{\G}{\mathbb{G}}
\newcommand{\Gm}{\G_m}
\newcommand{\Q}{\mathbb{Q}}
%\newcommand{\bP}{\mathbb{P}}
% \newcommand{\HH}{\mathbb{H}}
% \newcommand{\bV}{\mathsf{V}}
% \newcommand{\bVt}{\widetilde{\mathsf{V}}}

\newcommand{\bmu}{\boldsymbol{\mu}} 
\newcommand{\bka}{\boldsymbol{\kappa}} 
\newcommand{\bPsi}{\boldsymbol{\Psi}} 
\newcommand{\bP}{\mathbb{P}}
\newcommand{\bT}{\mathsf{T}}
\newcommand{\bO}{\mathsf{O}}
\newcommand{\bS}{\mathsf{S}}
\newcommand{\bA}{\mathsf{A}}
\newcommand{\bB}{\mathsf{B}}
\newcommand{\bC}{\mathsf{C}}
\newcommand{\cK}{\mathscr{K}}
\newcommand{\cA}{\mathscr{A}}
\newcommand{\cE}{\mathscr{E}}
\newcommand{\cB}{\mathscr{B}}
\newcommand{\cL}{\mathscr{L}}
\newcommand{\cO}{\mathscr{O}}
\newcommand{\cT}{\mathscr{T}}
\newcommand{\cS}{\mathscr{S}}
\newcommand{\cD}{\mathscr{D}}
\newcommand{\cV}{\mathscr{V}}
\newcommand{\cU}{\mathscr{U}}
\newcommand{\cG}{\mathscr{G}}
\newcommand{\cX}{\mathscr{X}}
\newcommand{\cY}{\mathscr{Y}}
\newcommand{\Mbar}{\overline{\mathscr{M}}}
\newcommand{\cXh}{\widehat{\cX}}
\newcommand{\Xh}{\widehat{X}}
\newcommand{\cW}{\mathscr{W}}
\newcommand{\bchi}{\boldsymbol{\chi}}
\newcommand{\bGa}{\boldsymbol{\Gamma}}
\newcommand{\lan}{\left\langle} 
\newcommand{\ran}{\right\rangle} 
\newcommand{\fw}{\mathfrak{w}}
\newcommand{\fC}{\mathfrak{C}}
\newcommand{\bw}{\mathbf{w}}
\newcommand{\bv}{\mathbf{v}}
\newcommand{\bs}{\mathbf{s}}
\newcommand{\bh}{\mathbf{h}}

\newcommand{\cUfgh}{\cU_\hbar(\widehat{\fg})}

%\newcommand{\fK}{\mathfrak{K}}
\newcommand{\fK}{\overline{Z}}

\newcommand{\task}[1]{\bigskip\noindent
\boxed{\texttt{#1}}\bigskip} 


\newcommand{\na}[1]{\nabla_{\! #1}}
\newcommand{\thbar}{\widetilde{\hbar}}
\newcommand{\bbV}{\mathbb{V}}
\newcommand{\bbM}{\mathbb{M}}
\newcommand{\tbM}{\widetilde{\bbM}}
\newcommand{\bbW}{\mathbb{W}}

\newcommand{\aroof}{\widehat{\mathsf{a}}}
\newcommand{\hX}{\widehat{X}}
\newcommand{\ve}{\varepsilon} 
\newcommand{\bMf}{\mathsf{M}^\mathbf{f}} 
\newcommand{\bMs}{\mathsf{M}^\mathbf{s}}

\newcommand{\crit}{\textup{crit}}


\newcommand{\Hd}{{H}^{\raisebox{0.5mm}{$\scriptscriptstyle \bullet$}}}
\newcommand{\Hsd}{{H}_{\raisebox{0.5mm}{$\scriptscriptstyle
      \bullet$}}}
\newcommand{\Ld}{{\Lambda}^{\raisebox{0.5mm}{$\scriptscriptstyle \bullet$}}}

\DeclareMathOperator{\Attr}{Attr}
\DeclareMathOperator{\qAttr}{\widehat{Attr}}
\DeclareMathOperator{\Db}{D^b}
\DeclareMathOperator{\Coh}{Coh}
\DeclareMathOperator{\Tor}{Tor}
\DeclareMathOperator{\Fr}{Fr}

\DeclareMathOperator{\Spec}{Spec}
\DeclareMathOperator{\supp}{supp}
\DeclareMathOperator{\rk}{rk}
\DeclareMathOperator{\wei}{weight}
\DeclareMathOperator{\virdim}{vir\ dim}


\DeclareMathOperator{\ch}{ch}
\DeclareMathOperator{\Def}{Def}
\DeclareMathOperator{\Obs}{Obs}
\DeclareMathOperator{\Ext}{Ext}
\DeclareMathOperator{\Lie}{Lie}
\DeclareMathOperator{\sgn}{sgn}
\DeclareMathOperator{\Res}{Res}
\DeclareMathOperator{\const}{const}
\DeclareMathOperator{\Ell}{Ell}
\DeclareMathOperator{\Gr}{Gr}
\DeclareMathOperator{\Mat}{Mat}
\DeclareMathOperator{\chr}{char}
\DeclareMathOperator{\Proj}{Proj}
\DeclareMathOperator{\Aut}{Aut}
\DeclareMathOperator{\Hom}{Hom}
\DeclareMathOperator{\End}{End}
\DeclareMathOperator{\Pic}{Pic}
\DeclareMathOperator{\pt}{pt}
\DeclareMathOperator{\tr}{tr}
\DeclareMathOperator{\ev}{ev}
\DeclareMathOperator{\Id}{Id}
\DeclareMathOperator{\Coker}{Coker}
\DeclareMathOperator{\Ker}{Ker}
\DeclareMathOperator{\ord}{ord}

\DeclareMathOperator{\Vx}{\mathbf{V}}
\DeclareMathOperator{\tVx}{\widetilde{\Vx}}
\DeclareMathOperator{\bVx}{\overline{\Vx}}
\newcommand{\be}{\mathbf{e}}
\newcommand{\bd}{\mathbf{d}}
\newcommand{\bla}{\boldsymbol{\lambda}}
\newcommand{\vth}{\vartheta} 
\newcommand{\vthb}{\overline{\vartheta}} 

\newcommand{\QM}{\mathsf{QM}}
\newcommand{\Ct}{\mathbb{C}^\times}

\newcommand{\Cp}{\mathsf{Cap}}
\newcommand{\Glue}{\mathsf{Glue}}
\newcommand{\Fusion}{\mathsf{Fusion}}
\newcommand{\Vertex}{\mathsf{Vertex}}
\newcommand{\tO}{\widehat{\mathscr{O}}}
\newcommand{\vir}{\textup{vir}}
\newcommand{\reg}{\textup{reg}}
\newcommand{\flop}{\textup{flop}}

\newcommand{\cF}{\mathscr{F}}
\newcommand{\tX}{\widetilde{X}} 
\newcommand{\tcF}{\widetilde{\cF}} 
\newcommand{\rd}{/\!\!/} 
\newcommand{\rdd}{/\!\!/\!\!/} 
\newcommand{\bfC}{\mathbf{C}}
\newcommand{\fh}{\mathfrak{h}} 
\newcommand{\fg}{\mathfrak{g}} 
\newcommand{\fgh}{\hat{\mathfrak{g}}} 
%\newcommand{\bM}{\overline{M}}
\newcommand{\MF}{\textsf{MF}}

\newcommand{\g}{\fg}

\DeclareMathOperator{\Stab}{Stab}
\DeclareMathOperator{\uStab}{\underline{Stab}}
\DeclareMathOperator{\catStab}{\underline{catStab}}
\DeclareMathOperator{\cStab}{hStab}
\DeclareMathOperator{\Hilb}{Hilb}
\DeclareMathOperator{\Conv}{Conv}
\DeclareMathOperator{\Cochar}{Cochar}
\DeclareMathOperator{\Image}{Im}
\DeclareMathOperator{\Crit}{Crit}
\DeclareMathOperator{\diag}{diag}




\newcommand{\hd}{{\displaystyle\boldsymbol{\cdot}}}
\newcommand{\eff}{\textup{eff}}
%\newcommand{\vir}{\textup{vir}}


\newcommand{\cMb}{\overline{\mathscr{M}}}
\newcommand{\cM}{\mathscr{M}}
\newcommand{\tphi}{\widetilde{\phi}} 
\newcommand{\tD}{\widetilde{\Delta}} 
\newcommand{\tM}{\widetilde{\cM}} 
\newcommand{\bbv}{\overline{\bv}} 
\newcommand{\bbw}{\overline{\bw}} 

\newcommand{\Xt}{X^{\sim}} 



\newtheorem{Proposition}{Proposition} 
\newtheorem{Lemma}{Lemma} 
\newtheorem{Corollary}{Corollary} 
\newtheorem{Theorem}{Theorem} 


\newcounter{mypoint}

\newcommand{\pnt}{\stepcounter{mypoint} \textbf{\arabic{mypoint}.} }


\newcommand{\question}{\marginpar{\quad \huge $\overset{\textup{?}}\leftarrow$}}

\newenvironment{NB}{
\color{red}{\bf NB}. \footnotesize
}{}
% The following is used inside NB
\newenvironment{NB2}{
\color{blue}{\bf NB2}. \footnotesize
}{}
%
\newenvironment{NB3}{
\color{purple}{\bf NB3}. \footnotesize
}{}
% For a final version, NB disappears.
% \excludeversion{NB}
% \excludeversion{NB2}
% \excludeversion{NB3}
\newcommand{\vin}[1]{{%\color{red}
\operatorname{i}(#1)}} % incoming vertex
\newcommand{\vout}[1]{{%\color{red}
\operatorname{o}(#1)}} % outgoing vertex

\begin{document}



\noindent
{\large \textsf{Action
    of shifted quantum loop groups on critical K-theory}} 
\medskip 
\hrule

\bigskip

\section{Overview}

\subsection{}


Let $X$ be a Nakajima quiver variety. The main goal of this project is to
construct and study a certain
action of a shifted quantum loop group on the
critical K-theory of the moduli spaces of stable quasimaps:
%
\begin{equation}
f: \bbA^1 \dasharrow X \,. \label{qmf}
\end{equation}
%

\subsection{}

Like ordinary quantum loop groups $\cUfgh$, the 
shifted quantum loop groups will be defined through matrix elements of
$R$-matrices with a spectral parameter. These $R$-matrices will
satisfy all relations of the ordinary quantum loop group and, in
particular, the Yang-Baxter equation. The only
difference will be in the behaviour of $R(u)$ as $u\to 0,\infty$.

Instead of the condition that $R(0)$ and $R(\infty)$ are finite and
invertible, we require that the image of
$$
R_{V_1,V_2}(u) \mapsto \End(V_1 \otimes V_2,\Bbbk[u^{\mp 1},u^{\pm 1}]])
$$
belong to certain strata that depend on the grading
by the weights of the Cartan subalgebra
$\fh \subset \fg$. The Cartan subalgebra $\fh$ acts on $V_i$ by
operators that record the ranks of the universal bundles, that is,
the dimension vectors for quiver varieties.

Singularities of $R$-matrices naturally appear in limit constructions
like infinite tensor products. The critical K-theory of quasimaps
moduli spaces will be related to such infinite tensor product
construction in what follows.

\subsection{}  

Let  $\cM$ be a smooth variety with a function 
%
\begin{equation}
\phi: \cM \to \bbA^1 \,.\label{W}
\end{equation}
%
We may assume that $0$ is the only critical value of $\phi$, that is,
that
$$
\Crit(\phi) = \{ d\phi =0\} \subset \phi^{-1}(0) \,. 
$$
By definition, 
$$
K_\crit(\phi) = \Coker \left(
K^\circ( \phi^{-1}(0)) \to K( \phi^{-1}(0)) \right)  
$$
is the cokernel of the natural map from the K-theory of locally free
sheaves to the K-theory of all coherent sheaves on the singular
fiber.

If $\phi$ is equivariant for an action of a group $G$ then an
equivariant version of $K_{\crit,G}(\phi)$ may be studied. This will be
very important to us and, in particular, we will make use of an action
that scales the target in \eqref{W} nontrivially. Perhaps it is a good
ideal to assume by default maximal possible equivariance in what
follows,
to save on
subscripts. \question 

As will be reviewed below, moduli spaces of stable quasimaps as in
\eqref{qmf} may be
presented as global critical loci. Thus the issue of gluing local data
does not arise in the construction of their critical K-theory.

\section{Equivariant critical K-theory}

\subsection{}

The inclusion
%
\begin{equation}
  \label{iota}
  \iota: \phi^{-1}(0) \to \cM 
\end{equation}
%
is a proper map to a smooth variety, and hence the pullback $\iota^*$
and the pushforward $\iota_*$ between the K-groups are well-defined.

We further assume that $\phi$ is equivariant for the action of a torus
$\bT$, which scales the target in \eqref{W} by a nontrivial character $w\in
\bT^\wedge$.  We denote by $\bT'\subset \bT$ the kernel of $w$. In general, this
may be a disconnected subgroup, but I don't know if we need this level
of generality. I believe that for the application we have in mind
$\bT'$ will be just a subtorus. \question

Also, I think we need to assume some equivariant formality about
$\cM$ \question 



\subsection{}

By localization, we have
%
\begin{equation}
  \iota_* \, \iota^* = (1- w^{-1}) \cdot \Id \label{loc1} \,. 
\end{equation}
%
In particular,
%
\begin{equation}
  \label{cokerii}
  \Coker \iota_* \, \iota^*  = K_{\bT'}(\cM) \,. 
\end{equation}

\subsection{}

We have
$$
\cM \setminus \phi^{-1}(0) \cong \left(\bT/\bT'\right) \times \phi^{-1}(1)
$$
and this identifies $\bT$-equivariant K-groups of $\cM \setminus
\phi^{-1}(0)$ with the $\bT'$-equivariant K-groups of $\phi^{-1}(1)$, pushed
forward under the inclusion $\bT'\to \bT$.

In particular, the long exact sequence of a pair becomes 
$$
\to K^1_{\bT'}(\phi^{-1}(1)) \to  K_\bT(\phi^{-1}(0))  \xrightarrow{\,\,\iota_*\,\,} K_\bT(\cM)
\to K_{\bT'}(\phi^{-1}(1)) \to 0 \,. 
$$
I believe $\phi^{-1}(1)$ is a variety of the same kind we always get, in
particular has no odd cohomology. So I think we have \question 
%
\begin{equation}
K^1_{\bT'}(\phi^{-1}(1)) = 0 \label{K1vanish}
\end{equation}
%
and hence an injection
%
\begin{equation}
  \label{injK1}
 0 \to  K_\bT(\phi^{-1}(0))  \xrightarrow{\,\,\iota_*\,\,} K_\bT(\cM) \,. 
\end{equation}

\subsection{}
On the other hand, every locally free sheaf on a subscheme comes from
one on the ambient space, hence dual surjection 
%
\begin{equation}
  \label{surjK}
\iota^*:  K(\cM) \to K^\circ( \phi^{-1}(0)) \to 0 \,. 
\end{equation} 
%

\subsection{}
Putting \eqref{cokerii}, \eqref{injK1}, and \eqref{surjK} together, we
obtain the following

\begin{Proposition}
  There is a natural injection
  %
\begin{equation}
 K_{\crit,\bT}(\phi) \hookrightarrow K_{\bT'}(\cM)\label{injKcrit} \,,
\end{equation}
%
in which the target is pushed forward with respect to $T'\hookrightarrow T$ and,
in particular, is a torsion $K_{\bT}(\pt)$-module. 
\end{Proposition}

It may be possible to imagine an abstract situation when the vanishing
\eqref{K1vanish} does not hold and so the injectivity of
\eqref{injKcrit} fails. I imagine this won't happen in situations of
interest to us, as this would give a nonzero canonical submodule of
$K_\crit$. 

\subsection{Perturbation/Degeneration}\label{s_pert} 
Now consider a potential $\tphi$ which is not $\bT$-equivariant 
and so has a nontrivial decomposition 
%
\begin{equation}
\tphi=\sum \tphi_i \label{tW}
\end{equation}
%
where the terms $\tphi_i$ have distinct $\bT$-weights $w_i$. Consider
the Newton polytope 
$$
\tD= \Conv(\{w_i\}) \in \bT^\wedge \otimes \R 
$$
of the function $\tphi$ and assume that it is contained in a hyperplane
of the form 
%
\begin{equation}
  \tD \subset \{x \, |\, \langle x, \eta \rangle = 1 \}\,, \quad
  \eta \in \Cochar(\bT) \otimes \Q \,. 
\end{equation}
%
Let $\bS\subset \bT$ be the torus with
$$
\Lie \bS = \R \eta + \bigcap \Ker dw_i \,. 
$$
By construction, the function $\tphi$ is $\bS$-equivariant with a
nontrivial weight. We denote by $\bS'\subset \bS$ the kernel of this
weight.

Let $\Delta\subset \tD$ be a face of some dimension and let
$$
\phi = \sum_{w_i \in \Delta} \tphi_i
$$
be the sum of corresponding terms in \eqref{tW}. 

\begin{Proposition}
  With the assumptions as above, there is a natural inclusion
  %
  \begin{equation}
    \label{pertW}
    K_{\crit,\bS}(\tphi) \subset K_{\crit,\bS}(\phi) \subset
    K_{\bS'}(\cM) \,. 
  \end{equation}
\end{Proposition}

\begin{proof}
  Let
  $$
  \tilde \iota: \tphi^{-1}(0) \to \cM
  $$
  be the inclusion. Since $\Delta$ is a facet of $\tD$,
  we can use the action of $\bT$ to contract $\tphi^{-1}(0)$ to
  $\phi^{-1}(0)$ in an $\bS$-equivariant way. Thus 
  $$
  \Image \tilde \iota_* \subset
  \Image \iota_* \subset K_{\bS}(\cM) \,.
  $$
 This implies the proposition. 
\end{proof}

\section{Examples of critical loci}

\subsection{Algebraic symplectic reductions}

\subsubsection{}\label{s_Mg}

Let $M$ be a symplectic representation of a group $G$ or a more
general smooth algebraic symplectic varieties on which a connected
reductive group $G$ acts in
a Hamiltonian way. Let
$$
\mu: M \to \fg^*\,, \quad \fg = \Lie G
$$
be the moment map. Consider a GIT quotient of the form
$$
\cM = \left(M \times \fg \right) \rd G  \,,
$$
about which we assume that it has no strictly semistable points.

\subsubsection{}

There is a natural function on $\cM$, namely 
%
\begin{equation}
\phi=\langle \mu(v), \xi \rangle \,, \quad (v,\xi) \in V \times \fg
\,.\label{phi_mu}
\end{equation}
%
We can take $\bT$ to be a maximal torus in the group 
$$
\bT \subset \Aut\left(M,\Bbbk \omega_M\right)^G \times \G_m
$$
of automorphisms of $\cM$ that act by transformations scaling the
the symplectic form on $M$ and dilations of the $\fg$-factor. The
function $\phi$ is invariant under a subtorus $\bT'\subset \bT$ that
scales the $\fg$-factor by
$$
\hbar = \textup{the weight of $\omega_M$} \, .
$$

\subsubsection{} 

Let $X$ be the critical locus of $\phi$. We have 
%
\begin{align}
  X = \Crit(\phi) &=
                    \{(v,\xi) \, |\,  \mu(v)=0, \xi\cdot v =0 \} \rd G\notag \\
 &= \mu^{-1}(0) \rd G \times \{0\} \label{xizero}
\end{align}
%
where the first line follows from the definition of the moment map,
and conclusion $\xi=0$ in the second line is forced by stability.

\subsubsection{} 


Since $\phi$ is linear in $\xi$ we have by the results of
\cite{}
$$
K_{\crit,\bT}(\phi) = K_{\bT'}(X) \,,
$$
thus equivariant K-theories of algebraic symplectic reductions are
examples of critical K-theories. 


\subsection{Jets of quasimaps to quiver varieties}
\label{sJets} 

\subsubsection{}

Now let $X$ be a Nakajima quiver variety which, by definition, means that
$X$ is an algebraic symplectic reduction of
$$
M = \bigoplus_{\textup{vertices $i$}} T^*\Hom(W_i,V_i)  \oplus
\bigoplus_{\textup{edges $i\to j$}} T^*\Hom(V_i,V_j)\,,
$$
by the group
%
\begin{equation}
G = \prod GL(V_i) \,.\label{GGV}
\end{equation}
%
Let $\Ct_\hbar$  scale the 
cotangent directions with weight $\hbar^{-1}$. 
The group 
$$
G_F = GL(W) \times \Ct_\hbar \times \dots\,, 
$$
where  $GL(W) = \prod GL(W_i)$, acts by 
 $G$-automorphisms of $M$ scaling the
symplectic form. 

\subsubsection{}


Now assume that the framing
space
$
W = \bigoplus_i W_i
$
has the form
%
\begin{equation}
W = \overline{W} \otimes \C^L \,,  \label{WbW}
\end{equation}
%
for some $L=2,3,\dots$. We restrict equivariance to the subgroup
%
\begin{equation}
  G_F \supset \overline{G}_F = GL(\overline{W}) \times \Ct_\hbar \times
  \dots \label{Gbar}
\end{equation}
%
in which $\Ct_\hbar$ also acts on $\C^L$ by
$$
\diag(1,\hbar,\hbar^2, \dots,\hbar^{L-1}) \in GL(\C^L) \,. 
$$
In particular, the operator
%
\begin{equation}
  \label{Xi}
  \Xi =
  \begin{pmatrix}
    0 \\
    1 &0  \\
    & 1 & 0 \\
    & & \ddots & \ddots 
  \end{pmatrix} \in \End(\C^L) \subset \End(W) 
\end{equation}
%
transforms with weight $\hbar$ under $\overline{G}_F$. 


In terms of the representation of quantum loop algebras, restriction to
$\overline{G}_F$ means
that all evaluation parameters split into $L$-tuples forming geometric
progressions with denominator $\hbar$.



\subsubsection{}

The function \eqref{phi_mu}  is linear in $\xi$ and has the form 
$$
\phi(A,B,\xi, \dots ) = \tr A \, B\, \xi + \dots
$$
where
$$
A \in \Hom(W,V) \,, \quad B\in \Hom(W,V)^* \cong \Hom(V,W) \,. 
$$
Define
$$
\tphi = \phi - \tr A \, \Xi \, B \,. 
$$
Since the addition is constant in $\xi$, this perturbation satisfies
the hypotheses of Section \ref{s_pert}.

\subsubsection{}\label{s_relW} 



The critical locus of $\tphi$ is the following. We have
$$
\frac{\partial}{\partial \xi} \, \tphi = \mu\,, \quad
\frac{\partial}{\partial A} \, \tphi  = B \xi - \Xi B \,, \quad
\frac{\partial}{\partial B} \, \tphi  = \xi A -  A \Xi \,, 
$$
while the partials with respect to coordinates in $\Hom(V_i,V_j)$ imply
that $\xi$ commutes with all these maps.

We can make $V$ and $W$ modules over $\C[x]$ by making $x$ act by
$\xi$ and $\Xi$ respectively. Then $\Crit(\tphi)$ is the locus where
the quiver maps are maps of $\C[x]$-modules, satisfying the moment map
equation. We note that
$$
W = \overline{W} \otimes \C[x]/x^L
$$
is a vector bundle on the $L$th infinitesimal neighborhood of $0 \in
\bbA^1$.

\subsubsection{}

To take the quotient by the group \eqref{GGV}, we need to impose a
certain stability condition. The most natural approach is to use the same stability condition as
in the construction of quiver varieties, that is, to use the GIT stability
corresponding to a character of $\prod GL(V_i)$.

If we do this,  the action of a quantum loop algebra on the
corresponding critical K-theories for finite $L$ should become
KR modules.

If, for large $L$, we want $V$ to similarly be an approximation to a
vector bundle on $\bbA^1$, then most likely we need to be more
selective about choosing the right stability condition. There is a
beginning of the discussion in Section \ref{s_limit} below. 
\question

\subsubsection{}

Let us show that the critical K-theories for finite $L$ with
$\overline{W} = \C$ supported at a single vertex are KR modules.

We specialize $\hbar$ generic, i.e., not a root of unity. The
corresponding specialized module is
$K_\crit(\tphi)\otimes_{R(\C^\times_\hbar)}\C$. It is isomorphic to
the complex critical K-group of the fixed point
$K_\crit(\tphi\big|_{\cM^{\C^\times_\hbar}})\otimes_\Z\C$. Connected
components of $\cM^{\C^\times_\hbar}$ correspond to gradings on $V$ as
for the usual quiver varieties. The direct sum decomposition into
K-groups of components is the $\ell$-weight space decomposition of the
module.

In order to show that it is irreducible, we show that there are no
$\ell$-dominant weight other than one corresponding to $V=0$. This
property is called \emph{special} \cite[Def.~10.1]{MR2144973} and
known to hold for KR modules \cite[Th.~3.2(1)]{MR1993360}. If it is
special, it is irreducible as a submodule should contains an
$\ell$-highest weight vector, which has an $\ell$-dominant weight. The
converse is not true in general.
%
\begin{NB}
The following is the original thought. It is kept as a record.
  
In turn, this should be true if we could see that $\Crit\tphi$ is
compact. This statement is known for ordinary quiver varieties.
\end{NB}%

In order to match the convention with the literature, we adjust the
$\C^\times_\hbar$-action so that $A$, $B$, $X$ are scaled by $\hbar$,
$\Xi$, $\xi$ are scaled by $\hbar^{-2}$, and hence the action on
$\C^L$ is $\diag(1,\hbar^{-2},\hbar^{-4}, \dots,\hbar^{2(1-L)})$.

The Drinfeld polynomial is calculated from the $\ell$-highest weight,
which corresponding to the component $V=0$. It should be clear, once
everything is settled.

Suppose that $\overline{W}$ is $\C$ supported at $i$. We take a fixed
point $\cM^{\C^\times_{\hbar}}$ and consider the corresponding weight
space decomposition $V = \bigoplus V_i(a)$, where $V_i(a)$ the weight
subspace of $V_i$ with character $a$. Let us define the weight space
$W_i(a)$ in the same way.
%
We have $W_i(1) = W_i(\hbar^{-2}) = \cdots = W_i(\hbar^{2(1-L)}) = \C$. On the
fixed point, $\xi$, $\Xi$ shift weights by $\hbar^{-2}$, 
%
$X$, $A$, $B$ shift by $\hbar$.

Assume $V_i(\hbar) = 0$. Then $A|_{W_i(1)} = 0$. The relation
$\xi A = A\Xi$ implies $A|_{W_i(\hbar^{-2})} = 0$ since
$\Xi\colon W_i(1)\to W_i(\hbar^{-2})$ is an isomorphism. Repeating the same
argument, we find $A = 0$ by induction. The stability condition
\begin{NB}
  Here the stability is the cyclic one.
\end{NB}%
implies that $V=0$. Therefore $V_i(\hbar)\neq 0$ if the fixed point
does \emph{not} correspond to the $\ell$-highest weight vector. Let us
assuming $V\neq 0$ hereafter. Let $V_j(\hbar^m)$ be a weight space
with the maximum power $m$. We have $m\ge 1$ by the above discussion.
%
Consider the following graded part of the complex
\begin{NB}
  I need to add a reference or explanation for the complex.
\end{NB}%
\begin{equation*}
  \begin{matrix}
    V_j(\hbar^m) &
    \xrightarrow{\left[\begin{smallmatrix}B\\ X_h\end{smallmatrix}\right]}
    & W_i(\hbar^{m+1}) \oplus \bigoplus_{h:\vout{h}=i} V_{\vin{h}}(\hbar^{m+1})
    & \xrightarrow{\left[\begin{smallmatrix} A & \pm X_{\overline{h}}
          \end{smallmatrix}\right]}
    & V_i(\hbar^{m+2}).
  \end{matrix}
\end{equation*}
By our choice of $m$ and the condition $m\ge 1$, the only nonzero term
in the complex is $V_j(\hbar^m)$. Therefore the alternating sum of
dimension
\begin{NB}
  where the middle is of degree $0$
\end{NB}%
is negative. Therefore the corresponding $\ell$-weight is \emph{not}
$\ell$-dominant. This shows that our module is special, hence is
irreducible.




In order to relate this with \cite{}, we probably need a Riemann-Roch
type theorem for critical theories, from K to homology.

It is also interesting to see the $T$-system
$$
0 \to \bigotimes_{j:a_{ij}=-1} W^{(j)}_{k,aq}
\to W^{(i)}_{k,a} \otimes W^{(i)}_{k,aq^2}
\to W^{(i)}_{k+1,a}\otimes W^{(i)}_{k-1,aq^2}\to 0
$$
in \cite[Th.~1.1(1)]{MR1993360} in this context.

\begin{NB}
An old example, where the convention is not fixed yet.
  
In type $A_1$, we have $W = \C[x]/x^L$ and have $A, B\colon W\to V$
are $\C[x]$-linear. Furthermore, if it corresponds to a fixed point,
$V$ is equipped with a grading, and $A$ preserves the gradings, $B$
decrease them by $1$. By the stability, $V = \C[x]/x^v$ and $v\le L$
such that $A$ is the natural projection
$\C[x]/x^L\twoheadrightarrow \C[x]/x^v$. Then $B$ sends $1$ to $0$,
then $B\xi = \Xi B$ implies that $B=0$. Thus the fixed point sets are
points, and compact.
\end{NB}%



\subsubsection{(outdated, I think)} 

{\footnotesize 
Suppose we want to study quasimaps to quiver variety
with a given dimension vector $\bbv$. 
I think the right stability condition might be something of the
following kind. It is convenient to think of the decomposition
\eqref{WbW} as follows in the style of Crawley-Boevey. Introduce
a new vertex $\infty$ with
$$
V_\infty = \C^L
$$
and the operator $\Xi$ acting in $V_\infty$. Further, connect
each vertex $i$ with $\infty$ by $\dim \overline{W}_i$ many edges.
Given a sufficiently generic stability condition
$$
\theta = (\theta_1,\theta_2, \dots)
$$
for the original quiver, extend it by setting 
$$
\theta_\infty = - \theta \cdot \bbv\,. 
$$
Let
$$
V' \subset \bigoplus_{i \in \textup{vertices} \cup \{\infty\}} V_i
$$
be a subspace invariant under the path algebra of the extended quiver
and the operators $\xi$ and $\Xi$. I imagine that a good condition to
impose on it is that 
$$
(\theta, \theta_\infty) \cdot (\dim V', \dim V'_\infty) \ge 0 
$$
for any such submodule.
}

% On the critical locus, when $(\xi,\Xi)$ commute with the quiver maps,
% this implies that the kernel of these operators is a stable
% representation of dimension $(\bv, 1)$, which is as close as we can
% get to $V$ being a torsion-free $\xi$-module. 




\section{Stable envelopes in critical K-theory}
\label{s_Stable}

\subsection{}

For any $\cM$ as in Section \ref{s_Mg}, categorical stable envelopes may be
constructed
as Fourier-Mukai functors
%
\begin{equation}
\Stab: \Db \Coh \cM^\bA  \to \Db \Coh \cM\,, \label{Stab1}
\end{equation}
%
for any torus 
$$
\bA \subset \Aut(M,\omega_M)^G
$$
and a choice of slope and attracting directions for $\bA$. These can be
shown be, in fact, sheaves, in some generality. Here we probably need only their K-theory
classes, that is, K-theoretic stable envelopes.

\subsection{}

Let $\phi$ be an $\bA$-invariant function on $\cM$. Consider the
attracting manifold 
$$
\Attr \subset \cM^\bA \times \cM 
$$
and let $p_1$ and $p_2$ be the two from the product. By construction
$$
\Attr \subset \big\{p_1^* \phi\big|_{\cM^\bA} - p_2^* \,
\phi = 0 \big\} \,. 
$$
We claim the same is true of the support of \eqref{Stab1}.

\subsection{}



We have $\supp \Stab \subset \Attr^f$, where $\Attr^f$ is the full attracting set, that is, the smallest
subscheme, closed under the operation of taking the attracting
manifold of $\bA$-fixed points. Let $R=\Gamma(\cO_\cM)$ denote
the algebra of global functions on $\cM$. 
We have the following diagram
%
\begin{equation}
  \xymatrix{
    \Attr^f  \ar[r] \ar[d] & \cM^\bA \times \cM \ar[d] \\
    \Attr_0  \ar[r]  & \cM_0^\bA \times \cM_0 
}\,,  \label{AttAtt}
\end{equation}
%
where
%
\begin{align}
 \cM_0&= \Spec R = \textup{the affinization of $\cM$} \\
  \Attr_0&= \Spec R/ \left(R_{>0}\right) \\ 
 \cM_0^\bA &= \Spec \left(R/ \left(R_{>0}\right) \right)^\bA  \,. 
\end{align}
%
Here $R_{>0}$ denote the subalgebra of functions of positive weight
with respect to a chosen cocharacter of $\bA$.

Since $\phi$ is pulled
back from the affinization, 
$$
\Attr^f \subset \big\{p_1^* \phi\big|_{\cM^\bA} - p_2^* \,
\phi = 0 \big\} \,. 
$$
Therefore, $\Stab$ induces a functor between the categories
of singularities, or the categories of matrix factorizations 
%
\begin{equation}
  \Stab: \MF(\phi\big|_{\cM^\bA} )  \to
  \MF(\phi) \,, \label{Stab2}
\end{equation}
%
and the corresponding Grothedieck groups, that is, critical
$K$-theories.

\subsection{}

In the same way, we have the restriction with support homomorphism
\begin{equation*}
  \iota^* \colon K_{\crit,\bT}(\phi)\to K_{\crit,\bT}(\phi\big|_{\cM^\bA})
\end{equation*}
with respect to the inclusion $\iota\colon \cM^\bA\to \cM$ of the
ambient smooth spaces. It becomes an isomorphism if we invert
$\bA$-weights of the normal bundle for $\cM^\bA\subset\cM$.

\subsection{}
Let $X$ be a Nakajima quiver variety and consider tori  of the form 
$$
\bA= \{ (a_1, a_2, \dots)\}, \quad a_i \in \Ct \,,
$$
acting on the framing spaces so that
%
\begin{equation}
W = a_1 W^{(1)} \oplus a_2 W^{(2)} \oplus \dots \,. \label{aWp}
\end{equation}
%
The fixed loci of such action are product of Nakajima quiver varieties
with the same quiver and framing $W^{(i)}$. 


Recall that equivariant K-theories of Nakajima quiver varieties are
made into a certain category of modules over a quantum loop group by
requiring that the stable envelopes (for fixed slope, and arbitrary
choice of attracting directions) are morphisms in this category for
all $\bA$ as above.

This morphisms are special cases of the maps \eqref{Stab2} for
the potential $\phi$ as in \eqref{phi_mu}.

\subsection{}
Perturbations $\tphi$ as in Section \ref{sJets} are
compatible with fixed points and
stable envelopes as long the operators like $\Xi$ are
$\bA$-invariant, that is, preserve the
direct sum decomposition in \eqref{aWp}.

Further, if $\prod X^{(i)}$ is a component of $X^\bA$ then
$$
\tphi\big|_{\prod X^{(i)}} = \sum \tphi_i
  $$
where each $\tphi_i$ is a function of the same form pulled back from
the corresponding factor. As a result
$$
K_\crit\left( \tphi\big|_{\prod X^{(i)}}  \right) = \bigotimes
K_\crit\left( \tphi_i\right)
$$
And this enlarges the previously constructed category of
modules by modules of the form $K_\crit\left( \tphi\right)$.
By the perturbation argument, these are submodules in
$K_\crit\left( \phi\right) \cong K(X)$.

In fact, we have a commutative diagram
$$
\begin{matrix}
K_\crit\left(\tphi\big|_{\cM^\bA}\right) & \xrightarrow{\Stab} & 
K_\crit\left(\tphi\right) \\
 \downarrow & & \downarrow \\
K_\crit\left(\phi\big|_{\cM^\bA}\right) & \xrightarrow[\Stab]{} & K_\crit\left(\phi\right),
\end{matrix}
$$
hence the $R$-matrrix for $K_\crit\left(\phi\right)$ preserves
$K_\crit\left(\tphi\right)$.

\section{Large $L$ limit}\label{s_limit} 

\subsection{The case $\dim V$ stays finite}

\subsubsection{}

Clearly, $\Xi^L=0$. Let
%
\begin{equation}
  V = V_{=0}\oplus V_{\ne 0} \,, \quad
  W= W_{=0} \oplus W_{\ne 0} \label{V0Vn}
\end{equation}
%
be the decomposition according to zero/nonzero eigenvalues of
the operators $(\xi,\Xi)$. The relations of Section \ref{s_relW}
show that this is a direct sum of quiver representations with
$W_{\ne 0}=0$. By stability, $V_{\ne 0}=0$, that is, $\xi$ is
nilpotent. Therefore
$$
\xi^{\dim V_i}\big|_{V_i} = 0
$$
meaning that $\xi^k = 0 $ for some fixed $k$ that does not grow with
$L$.

\subsubsection{}

This means that
$$
\Image B \subset \Ker \Xi^k \subset
\Image \Xi^k \subset \Ker A \,, 
$$
for all $L\ge 2k$. It follows that $AB=0$ and therefore both $A$
and $B$ cancel out of the moment map equations.

Some of them are still
needed to insure stability, the rest becomes a vector bundle over a
simpler moduli space.

\subsubsection{}

For instance, suppose the stability condition is such that the quiver
maps and $\xi$ generate the spaces $V$ from the image of $A$.
Clearly, it is enough to take
$$
\overline{A}: \overline{W} \to V
$$
where $\overline{W} \subset W$ is embedded as the trivial
weight space of $\hbar$. Similarly, $B$ is uniquely determined by the composition
$$
\overline{B}: V \xrightarrow{\, B\, } W \to \overline{W} \otimes
h^{L-1} 
$$
where the projection is to the corresponding $\hbar$-eigenspace.

While $\overline{A}$ plays a role in stability, the datum of
$\overline{B}$ is completely arbitrary and forms a vector
bundle over the moduli space of singly-framed representations.

For
instance, if the original quiver variety was $\Hilb(\C^2)$, we will
get the total space of a tautological bundle over $\Hilb(\C^3)$, where
$\xi$ acts as the multiplication by the new coordinate.


\subsection{When $V$ grows with $L$: singularities in the $R$-matrix}

\subsubsection{}

In the notation of Section
\ref{sJets}, we fix the dimension vector
$$
\bbw= \dim \overline{W}\,, 
$$
and consider the direct sum of the corresponding
critical K-theories 
$$
\bbM = \bigoplus_{\bv} K_\crit(\tphi)_{\bv,\bw}\,, \quad
\bw = L\bbw\,, 
$$
over all possible dimension vectors $\bv$.

By the results of Section \ref{s_Stable}, this is a module over a
certain quantum loop group. Concretely, its structure may be
described by $R$ matrices $R_{\bbM',\bbM}(u)$ with
spectral parameter $u$, where
$$
\bbM' = \bigoplus_{\bv'} K_\bT(X_{\bv',\bw'})\,, 
$$
and $X_{\bv',\bw'}$ denotes the Nakajima quiver variety with
dimension vectors $\bv'$ and $\bw'$.

For fixed $L$ these $R$ matrices are rational functions of $u\in \Ct$ (and
all other parameters in the theory) such that
$R(0)$ and $R(\infty)$ are finite and invertible.
We will see how singularities at
$u=0,\infty$ appear in the $L\to \infty$ limit.

\subsubsection{}
The singularities of the $R$-matrix are best described in terms of its
Gauss decomposition given by the stable envelopes, see
\cite{} for an introduction. In fact, the
singularities appear precisely in the diagonal part of the Gauss
decomposition.

\subsubsection{} 

Let $F$ be a component of the fixed locus $X^\bA$ for an action of a
torus $\bA \subset \Aut(X,\omega_X)$. The normal bundle to
$F$ splits into the attracting and repelling directions 
$$
N_{X/F} = N_{>0} \oplus N_{<0}
$$
for a choice of a generic cocharacter of $\bA$. By construction
of the stable envelopes, the diagonal part of the Gauss decomposition
of the corresponding $R$-matrix is given by
$$
D_F = \aroof\left(N_{<0} -\hbar^{-1}\, N_{<0} \right) \,. 
$$
Here $\aroof$  is the multiplicative genus defined by
$$
\aroof (\cL) = \cL^{1/2} - \cL^{-1/2}
$$
for a line bundle $\cL$.

\subsubsection{} \label{s_LL} 

Now suppose that
$$
N_{<0}  = \overline{N_{<0}} \otimes (1+\hbar+\dots+\hbar^{L-1}) +
\delta N_{<0} \,,
$$
where the second term is fixed in the $L\to \infty$ limit. Set
$$
D'_F = \aroof\left(-\hbar^{-1}\overline{N_{<0}} +(1-\hbar) \delta
  N_{<0} \right) \,.
$$
By telescoping, we get the following

\begin{Lemma}
  \begin{equation}
    \label{limD_F}
    \frac{D_F}{D'_F} \sim
    \begin{cases}
      \left(\det \hbar^{L-1} \overline{N_{<0}} \right)^{1/2} \,, &
      \hbar^{L} \to \infty\,, 
      \\
      (-1)^{\rk \overline{N_{<0}}} \left(
        \det \hbar^{L-1} \overline{N_{<0}} \right)^{-1/2} \,, &
      \hbar^{L} \to 0\,. 
    \end{cases}
  \end{equation}
%
\end{Lemma}

\subsubsection{}

Concretely, in our situation, $F$ is contained in the locus of
direct sums $Q'\oplus Q$
of quiver representations with
$$
\dim Q' = (\bw',\bv')\,, \quad \dim Q = (\bw,\bv) \,. 
$$
The direct sums are fixed under the action of the torus
$$
\bA \cong \Ct \owns u
$$
that scales the $W'$ framing spaces with weight $u$. If we declare the
weight $u$ to he attracting, then
$$
N_{<0} = u^{-1} \Ext^1(Q',Q) \,.
$$
In particular,
%
\begin{equation}
  \label{rkN<}
  \rk N_{<0} = \bB((\bv',\bw'), (\bv,\bw)) 
\end{equation}
%
where
%
\begin{equation}
  \label{Bvv}
  \bB((\bv',\bw'), (\bv,\bw))  = \bv' \cdot \bw + \bw' \cdot \bv -
  \bv' \cdot \bC \bv\,. 
\end{equation}
%
Here $\bC$ is the Cartan matrix of the quiver.


\subsubsection{}

We consider the case when the dimension of $Q'$ is fixed, while the
dimension of $Q$ grows with $L$ so that
$$
\bw = L \, \bbw \,, \quad \bv = L \, \bbv + O(1) \,. 
$$
With the correct stability condition, we have 
$$
[Q] = (1+\hbar+\dots + \hbar^{L-1}) [\overline{Q}] +
[\delta Q]
$$
in the K-theory of quiver representations, where $[\delta Q]$ is
bounded K-theory class. This puts us in the situation of Section
\ref{s_LL} with
$$
\overline{N_{<0}} = u^{-1} \Ext^1(Q', \overline{Q}) \,, 
$$
and gives the following

\begin{Proposition}
  Let the weight $u$ be attractive and let
  $$
  \overline{D}_{Q'\oplus
    Q} = \lim_{L\to \infty}  h^{\dots} D_{Q'\oplus
    Q}
  $$
  denote the leading term in the $L\to \infty$ asymptotics
  of the diagonal diagonal part in the Gauss decomposition of the
  $R$-matrix. If $\hbar^{L}\to \infty$ then
  %
  \begin{equation}
    \label{eq:3}
    \ord_{u=0} \overline{D}_{Q'\oplus
    Q} = 0 \,, \quad \ord_{u=\infty} \overline{D}_{Q'\oplus
    Q} = \bB((\bv',\bw'), (\bbv,\bbw)) \,. 
\end{equation}
 If $\hbar^{L}\to 0$ then
  %
  \begin{equation}
    \label{eq:3}
    \ord_{u=0} \overline{D}_{Q'\oplus
    Q} =  \bB((\bv',\bw'), (\bbv,\bbw)) \,, \quad \ord_{u=\infty} \overline{D}_{Q'\oplus
    Q} = 0 \,. 
\end{equation}
\end{Proposition}

\noindent
Here $ \ord_{u=0}$ denotes the order of vanishing at $u=0$.





















\end{document}


%%% Local Variables:
%%% mode: latex
%%% TeX-master: t
%%% End:
