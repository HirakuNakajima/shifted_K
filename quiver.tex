\section{}

\subsection{Quiver description of the affine Grassmannian}

\subsubsection{}

Let $\cK:= \C((x))$, $\cO := \C[[x]]$.
%
Let $\oW$ be a finite dimensional complex vector space.

Recall that the affine Grassmannian $\Gr_{G}$ for $G = \GL(\oW)$ is
the space of lattices $\cL$ in $\oW_\cK = \oW\otimes_\C \cK$. Such a
lattice $\cL$ is between $x^{L/2}\oW_\cO \subset x^{-L/2} \oW_\cO$ for
sufficiently large $L$. We assign the following data to $\cL$:
\begin{gather*}
    V := x^{-L/2} \oW_\cO/\cL \ \text{(a finite dimensional vector space)}, \\
    A \in \Hom(\oW, V);\ A(w) := w\otimes x^{-L/2} \mod \cL,\\
    \xi \in \Hom(V,V);\ \xi := \text{the multiplication by $x$}.
\end{gather*}
From the construction $A$ is cyclic with respect to $\xi$, namely a
subspace of $V$ containing $\Image A$ and stable under $\xi$ must be
the whole $V$.

Conversely we can recover $\cL$ from $V$, $A$, $\xi$ as
\begin{equation*}
  \cL := \left\{ w\otimes x^{-L/2} f\in x^{-L/2} \oW_\cO
  \middle| f(\xi) A(w) = 0 \right\}.
\end{equation*}

For example, the identity element in $\Gr_G$, which is
$\cL = \oW_\cO$, corresponds to
\begin{gather*}
  V = x^{-L/2}\oW_\cO/\oW_\cO, \quad
  A(w) = w\otimes x^{-L/2} \mod \oW_\cO, \\
  \xi = \text{the multiplication by $x$}.
\end{gather*}

This gives us a finite dimensional approximation of the affine
Grassmannian by the GIT quotient
\begin{equation*}
  \cQ(V,\oW) \equiv \cQ :=
   \Hom(\oW,V) \oplus \gl(V)^{\mathrm{nilp}} \rd_{\!\mathrm{cyc}} \GL(V),
\end{equation*}
where `nilp' means that nilpotent endomorphisms, `cyc' means the
cyclicity condition above, and is given by the character
$\det\colon \GL(V)\to \C^\times$.
\begin{NB}
  We have $\dim \cQ = (\dim \oW - 1)\dim V$.
\end{NB}%
 
Lusztig identified the nilpotent cone for $\gl(\oW)$ with a locally
closed subset of $\Gr_{G}$ \cite{Lu-Green}. It is the open
subvariety of $\cQ$ for $\dim V= \dim \oW$ consisting of invertible
$A$'s. Then we normalize $A$ to $\Id$ to kill the action of
$\GL(V)$. Therefore the open subset is indeed the nilpotent cone.

The notation for $\cQ(V,\oW)$ follows the usual convention: a vector
space whose $\GL$-quotient appears is denoted by $V$, and another
(framed) vector space by $\oW$. However we use the affine Grassmannian
for $\Gr_{\GL(V)}$ for a quiver gauge theory associated with $V$, $\oW$
in the context of Coulomb branches. Therefore we need to change the
notation for Coulomb branches afterwards, namely $\oW$ here is the usual
$V$.

\begin{NB}
I am not sure the following is necessary:
  
We can alternatively consider
\begin{gather*}
  V' := \cL / x^{L/2} \oW_\cO\ \text{(a finite dimensional vector space)}, \\
  B \in \Hom(V', \oW);\ B(v) := \frac1{2\pi i}\oint v x^{-L/2} dx,\\
    \xi \in \Hom(V',V');\ \xi := \text{the multiplication by $x$},
\end{gather*}
where the contour is a small circle around $x=0$ in the anti-clockwise
direction. Then $(B,\xi)$ is cocyclic, i.e.\ a subspace in $V'$
invariant under $\xi$ and sent to $0$ by $B$ must be $0$.

Let us consider the perpendicular $\cL^\perp$ of $\cL$ in
$\oW^*_\cK$. Then $\cL^\perp\subset x^{-N} \oW^*[[x]]$ and the
previous construction gives
$V^{\prime*} = x^{-N} \oW^*[[x]]/\cL^\perp$ and
$B^*\in\Hom(\oW^*,V^{\prime*})$,
$\xi\in\Hom(V^{\prime*},V^{\prime*}) = \Hom(V',V')$. Taking
transposes, we get the above description.
\end{NB}%

\subsubsection{}\label{subsubsec:alt}

We introduce $W = \oW\otimes_\C x^{-L/2} \C[x]/x^{L/2}\C[x]
= \oW\otimes_\C x^{-L/2} \cO/x^{L/2} \cO$,
and $\Xi\in\End(W)$ as the multiplication by $x$ as in
\cref{sJets}. Then $A$ extends uniquely to $W$ so
that $A \Xi = \xi A$. It becomes surjective. Therefore
\begin{equation*}
  \cQ \cong \left\{ (A,\xi)\in\Hom(W,V)\oplus\gl(V) \middle|
  A\Xi = \xi A\right\}\rd_{\!\mathrm{surj}} \GL(V),
\end{equation*}
where `surj' means that $A$ is surjective. The nilpotency of $\xi$ is
automatically satisfied as $\xi^n A = A\xi^n = 0$ for $n\ge \dim W$.

\subsubsection{}

The positive affine Grassmannian $\Gr_G^+$ consists of $\cL$ such that
$\cL\subset \oW_\cO$. Its finite dimensional approximation can be
defined as
\begin{equation*}
  \cQ^+ := \left\{ (A^+,\xi^+)\in\Hom (W^+, V^+)\otimes\gl(V^+)
    \middle| A^+\Xi = \xi^+ A^+\right\}\rd_{\!\mathrm{surj}} \GL(V^+),
\end{equation*}
where $W^+ = \oW_\cO/x^{L/2} \oW_\cO$, $V^+ = \oW_\cO/\cL$,
$A^+(w) = w\otimes 1 \mod \cL$, and $\xi$, $\Xi$ are multiplication by
$x$.

The embedding $\cQ^+\subset\cQ$ is given as follows: let
$W^- = x^{-L/2} \oW_\cO /\oW_\cO$. We have $\xi^-\in\gl(V^-)$ by the
multplication of $x$, and $\xi'\colon W^-\to W^+$ by
$w\otimes x^{-1} \to w\otimes 1 \mod \oW_\cO$, $w\otimes x^{-n}\to 0$
for $n\neq -1$. Then we set
\begin{equation*}
   V = W^-\oplus V^+,\ W = W^-\oplus W^+,\ A =
   \begin{bmatrix}
     \Id_{W^-} & A^+
   \end{bmatrix},\ 
   \xi =
   \begin{bmatrix}
     \xi^- & 0 \\
     A^+\xi'  & \xi^+
   \end{bmatrix}.
\end{equation*}


\subsubsection{}\label{subsubsec:ind}

If $\cL$ is contained in $x^{-L/2} \oW_\cO$, it is also contained in
$x^{-(L+2)/2} \oW_\cO$. If $\cL$ gives $(V,A,\xi)$ for $x^{-L/2} \oW_\cO$, the
quiver data for $x^{-(L+2)/2} \oW_\cO$ is $(V',A',\xi')$ with
\begin{equation*}
   V' := V\oplus \oW, \quad A' := %\left[
     \begin{bmatrix}
        0 \\ \Id_{\oW}
     \end{bmatrix}, \quad
     \xi' :=
     \begin{bmatrix}
       \xi & A \\
       0 & 0
     \end{bmatrix}.
\end{equation*}
We thus have the inductive system
\begin{equation*}
  \cQ(V,\oW) \subset \cQ(V\oplus \oW, \oW) \subset
  \cQ(V\oplus \oW^{\oplus 2}, \oW) \subset \cdots.
\end{equation*}

In the description in \cref{subsubsec:alt} we increase
$W = \oW\otimes_\C x^{-L/2}\C[x]/x^{L/2}\C[x]$ as $L\to \infty$. We
then simultaneously increase $V$ as $V$, $V\oplus \oW$,
$V\oplus\oW^{\oplus 2}, \cdots$. Thus $\dim V = L \dim \oW/2 + O(1)$.

\subsubsection{}

We have an action of $G_\cO$ on $\Gr_{G}$ induced from its
action on $\oW_\cK$. It is given by
\begin{equation*}
  A \mapsto A g(\Xi)^{-1} = A g_0 + \xi A g_1 + \xi^2 A g_2 + \cdots, \qquad
  \xi \mapsto \xi,
\end{equation*}
where $g(x) = (g_0 + g_1x + g_2 x^2 + \cdots)^{-1} \in G_\cO$. In the
first expression $A g(\Xi)^{-1}$, we regard $A$ as $\Hom(W,V)$. We
then use $A\Xi = \xi A$ to change it to the second expression which
makes sense for $A\in\Hom(\oW,V)$.
%
Note that it descends to an action of a finite
dimensional quotient $G(\cO/x^{\dim V}\cO)$.

This action is compatible with the inductive system in
\cref{subsubsec:ind} as
\begin{equation*}
  \begin{split}
    & A' g_0 + \xi' A' g_1 + (\xi')^2 A' g_2 + \cdots
    = \tilde g
    % \begin{bmatrix}
    %   \Id_V & A g_1 + \xi A g_2 + \cdots \\
    %   0 & g_0
    % \end{bmatrix}
    \begin{bmatrix}
      0 \\ \Id_{\oW}
    \end{bmatrix},\\
    &
    \begin{bmatrix}
      \xi & A \\
      0 & 0
    \end{bmatrix}
    = \tilde g
    % \begin{bmatrix}
    %   \Id_V & A g_1 + \xi A g_2 + \cdots \\
    %   0 & g_0
    % \end{bmatrix}
    \begin{bmatrix}
      \xi & A g_0 + \xi A g_1 + \xi^2 A g_2 + \cdots \\
      0 & 0
    \end{bmatrix}
    \tilde g^{-1},
    % \begin{bmatrix}
    %   \Id_V & A g_1 + \xi A g_2 + \cdots \\
    %   0 & g_0
    % \end{bmatrix}^{-1}.
    \\
    & \quad\text{where }\tilde g =
    \begin{bmatrix}
      \Id_V & A g_1 + \xi A g_2 + \cdots \\
      0 & g_0
    \end{bmatrix}.
  \end{split}
\end{equation*}

The loop rotation $\C^\times_{\mathrm{loop}}$ acts on $\cO$, $\cK$ by
$x\mapsto tx$ ($t\in\C^\times_{\mathrm{loop}}$). It induces an action
on $\Gr_{G}$, and is given by
\begin{equation*}
  A \mapsto A, \qquad \xi \mapsto t\xi
\end{equation*}
in the quiver description. This action is also compatible with the
inductive system in \cref{subsubsec:ind} as
\begin{equation*}
  \begin{bmatrix}
    0 \\ \Id_\oW
  \end{bmatrix} =
    \begin{bmatrix}
    t & 0\\
    0 & 1
  \end{bmatrix}
  \begin{bmatrix}
    0 \\ \Id_\oW
  \end{bmatrix}, \quad
  t \begin{bmatrix}
    \xi & A \\
    0 & 0
  \end{bmatrix}
  =
  \begin{bmatrix}
    t & 0\\
    0 & 1
  \end{bmatrix}
  \begin{bmatrix}
    t\xi & A \\
    0 & 0
  \end{bmatrix}
  \begin{bmatrix}
    t & 0\\
    0 & 1
  \end{bmatrix}^{-1}.
\end{equation*}

\subsubsection{}

Let us take a base of $\oW$ and consider the torus $T(\oW)\subset G$
consisting of diagonal matrices. It acts on $\Gr_{G}$ as a
subgroup of $G_\cO$. If $[A, \xi]$ is a fixed point, we have a
decomposition $V = V_1\oplus \cdots \oplus V_w$ ($w=\dim \oW$) which is
preserved by $\xi$. And $\prod_{i=1}^w [A_i,\xi_i]$ can be regarded as
a point in $\prod_{i=1}^w \cQ(V_i,\C)$, the quiver description of
$\Gr_{T(\oW)}$. Each $[A_i,\xi_i]$ is determined uniquely by $\dim V_i$,
and corresponds to the point $x^{\dim V_i - L/2}\cO \subset \cO$ in the
affine Grassmannian for $\C^\times$. Thus $\prod_{i=1}^w [A_i,\xi_i]$
is identified with the cocharacter $(\dim V_1-L/2,\cdots,\dim V_w-L/2)$ of
$G$.

\begin{NB}
By the localization in equivariant Borel-Moore homology groups, we
have
\begin{equation*}
  H^{T(\oW)}_*(\cQ)\otimes_{H^*_{T(\oW)}(\mathrm{pt})}
  \mathrm{Frac}(H^*_{T(\oW)}(\mathrm{pt}))
  \cong H_*(\cQ^{T(\oW)})\otimes_\C
  % \bigoplus_{\sum \dim V_i = \dim V}
  \mathrm{Frac}(H^*_{T(\oW)}(\mathrm{pt})).
\end{equation*}
\end{NB}%

\subsubsection{}\label{subsubsec:convolution}

Recall the convolution diagram appeared in the geometric Satake
correspondence (see e.g.\ \cite{MV2}):
\begin{equation*}
  \Gr_{G} \times \Gr_{G} \xleftarrow{p}
  G_\cK \times \Gr_{G} \xrightarrow{q}
  G_\cK \times_{G_\cO} \Gr_{G} \xrightarrow{m}
  \Gr_{G}.
\end{equation*}
In terms of lattices, the second space $G_\cK \times \Gr_{G}$
parametrizes $(\cL_1,\kappa,\cL_2)$ where $\cL_1$, $\cL_2$ are
lattices in $\oW_\cK$, and $\kappa$ is an isomorphism
$\cL_1\xrightarrow{\cong} \oW_\cO$. The third space $G_\cK \times_{G_\cO} \Gr_{G}$
parametrizes $(\cL_1,\cL_3,\eta)$ where $\cL_1$ is a lattice in
$\oW_\cK$, $\cL_3$ is an abstract projective $\cO$-module of rank
$=\dim \oW$, and $\eta$ is an isomorphism
$\cL_1\otimes_\cO\cK\cong \cL_3\otimes_\cO\cK$.
%
The map $p$ forgets $\kappa$. The map $q$ is given by setting
$\cL_3 = \kappa^{-1}(\cL_2)$, $\eta$ is induced from the natural inclusion
$\cL_3\subset \cL_1\otimes_\cO \cK$.
%
For the map $m$, we first define the embedding $\cL_3\subset \oW_\cK$
as the composite of $\eta^{-1}$ and
$\cL_1\otimes_\cO\cK\xrightarrow{\cong} \oW_\cK$. Then we consider
$\cL_3$ as a lattice in $\oW_\cK$.

Let us give a quiver description of the convolution diagram. We take
finite dimensional vector spaces $V_1$, $V_2$ to describe
$\Gr_G\times \Gr_G$. We also take a vector space $V_3$ with
$\dim V_3 = \dim V_1+\dim V_2$. It is for the rightmost $\Gr_G$.

We consider the space
\begin{equation*}
  \{ (V_3,A_3,\xi_3,\varphi\colon V_3\twoheadrightarrow V_1) \}/\!\cong,
\end{equation*}
where $V_3,A_3,\xi_3$ are as above, and $\varphi$ is a surjective
$\cO$-module homomorphism, i.e.\ a surjective linear map whose kernel
is invariant under $\xi_3$. We claim that this space is a finite
dimensional approximation of $G_\cK\times_{G_\cO}\Gr_G$ such that $m$
is given by forgetting $\varphi$.
%
In fact, we first define $\cL_3\subset x^{-L}\oW_\cO$ from
$(V_3,A_3,\xi_3)$ as above. We then define $A_1$ as the composite of
$A_3$ and the projection, $\xi_1$ as the multiplication by $x$. They
form data for $\cQ(V_1,\oW)$, hence we have the corresponding lattice
$\cL_1\subset x^{-L/2} \oW_\cO$. The surjection
$\varphi\colon V_3\twoheadrightarrow V_1$ gives an inclusion
$\cL_3\subset x^{-L/2} \cL_1$ as
$x^{-L}\oW_\cO/\cL_3 = V_3 \twoheadrightarrow V_1 = x^{-L/2} \oW_\cO/\cL_1
= x^{-L} \oW_\cO/ x^{-L/2} \cL_1$.
%
We choose an isomorphism $\kappa\colon \cL_1\xrightarrow{\cong} \oW_\cO$
in the second space $G_\cK\times\Gr_G$. Then we have an inclusion
$\cL_3\subset x^{-L/2}\cL_1\xrightarrow{\kappa} x^{-L/2} \oW_\cO$. Hence it
defines data
$V_2 = \Ker\varphi = x^{-L/2}\cL_1/\cL_3 \xrightarrow[\cong]{\kappa}
x^{-L/2} \oW_\cO/\kappa(\cL_3)$ and $A_2$, $\xi_2$ as before. This defines the
map $p$.

\subsubsection{}

We consider $K_{G_\cO\rtimes\C^\times_{\text{loop}}}(\Gr_G)$ and
define the convolution product by
\begin{equation*}
  c_1\ast c_2 = m_* (q^*)^{-1} p^* (c_1\otimes c_2) \quad
  \text{for $c_1$, $c_2\in K_{G_\cO\rtimes\C^\times_{\text{loop}}}(\Gr_G)$}.
\end{equation*}
Here $q$ is a $G_\cO$-bundle, hence $q^*$ is an isomorphism
\begin{equation*}
  K_{G_{\cO}\rtimes\C^\times_{\text{loop}}}(G_\cK\times_{G_\cO}\Gr_G)
    \xrightarrow[\cong]{q^*}
    K_{(G_{\cO}\times G_\cO)\rtimes\C^\times_{\text{loop}}}
    (G_\cK\times\Gr_G).
\end{equation*}
This is the quantized Coulomb branch associated with the group
$G = \GL(\oW)$ together with $0$ representation
\cite{2016arXiv160103586B}. In fact, this example was studied earlier
in \cite{MR2135527}.

\subsubsection{}

We embed $\cQ$ into a larger space
\begin{equation*}
  \cM := \Hom(W,V) \oplus \Hom(V,W)\oplus
  \gl(V) \rd_{\!\mathrm{cyc}} \GL(V),
\end{equation*}
where we denote the second component by $B$. Here
$W = \oW\otimes_\C \C[x]/x^{L}$, and $\Xi$ is the multiplication of
$x$ as before. This is an example of spaces considered in
\cref{Asr}. We define a function $\tphi$ as
$\tr(AB \xi) - \tr(A\Xi B)$ as before.
%
We let $G_\cO$ act on $B$ by $g(\Xi) B$ for $g(x)\in G_\cO$. Then
$\tphi$ is invariant under $G_\cO$. Hence we can consider
$G_\cO$-equivariant critical K-theory. Since $\tphi$ is linear in $B$, we
can reduce the critical K-theory to the ordinary K-theory on the variety
$0 = \partial\tphi/\partial B = \xi A - A\Xi$. In particular, we have
$\xi^L A = 0$, and hence $\xi^L = 0$ as $A$ is surjective. Thus we have
\begin{equation*}
  K_{\crit,?}(\tphi) = K_{?}(\cQ).
\end{equation*}
I have used this isomorphism when $\cQ$ is not smooth. Is it OK ? The
section $\xi A - A\Xi\in\Gamma(\Hom(W,V))$ is \emph{not} regular, as
$\dim(\text{zero set}) = \dim \cQ = (\dim \oW-1)\dim V > 0$ unless
$\dim\oW = 1$. \question

Alternatively we may consider
\begin{equation*}
  \bar{\cM} := \Hom(\oW,V) \oplus \Hom(V,\oW)\oplus
  \gl(V) \rd_{\!\mathrm{cyc}} \GL(V).
\end{equation*}
We consider a slightly modified function $\phi = \tr(A B \xi^d)$ for a
positive integer $d$. In fact, this is the function which was used by
Bykov, Zinn-Justin \cite{2019arXiv190411107B}.

We let $G_\cO$ act on $B$ by
\begin{equation*}
  B\mapsto g'_0 B + g'_1 B\xi + g'_2 B\xi^2 + \cdots
\end{equation*}
where $g(x) = g'_0 + g_1' x + g'_2 + \cdots$. Then
$\tr(A B \xi^d)$ is invariant under $G_\cO$.
\begin{NB}
  If we extend $B$ to $\Hom(V,W)$, it is clear. Otherwise we check it
  directly as
\begin{equation*}
  \begin{split}
    &
    \tr(A B \xi^d) \mapsto \\
    & \quad
    \begin{aligned}[t]
      & \tr\bigl( A g_0 g'_0 B \xi^d + \xi A g_1 g'_0 B \xi^d + A g_0
      g'_1 B \xi^{d+1}\\
      & \qquad
      + \xi^2 A g_2 g'_0 B\xi^d + \xi A g_1 g'_1 B\xi^{d+1}
      + A g_0 g'_2 B\xi^{d+2} + \cdots \bigr) = \tr(A B \xi^d)
    \end{aligned}
  \end{split}
\end{equation*}
as
$g(x)^{-1} g(x) = g_0 g'_0 + (g_1 g'_0 + g_0 g'_1) x + (g_2 g'_0 + g_1
g'_1 + g_0 g'_2)x^2 + \cdots = 1$. Thus $\phi$ is invariant under
$G_\cO$.
\end{NB}%
Hence we can consider the $G_\cO$-equivariant critical K-theory. We
have
\begin{equation*}
  K_{\crit,?}(\phi) = K_{?}(\widetilde\cQ),
\end{equation*}
where $\widetilde \cQ$ is the space defined by $\xi^d A = 0$ instead
of imposing that $\xi$ is nilpotent. But the cyclicity implies that
$\xi^d = 0$. Hence $\widetilde\cQ\subset \cQ$ and we have an equality
if we take $d \ge \dim V$.

\subsubsection{}

By \cref{subsec:Vgrow} we have an action of a shited quantum affine
algebra of type $\mathfrak{sl}(2)$ on $\bigoplus
K_{\crit,?}(\tphi)$. On the other hand, we have a surjective
homomorphism from a shifted quantum affine algebra to
$\lim_{L\to\infty} K_{G_\cO\rtimes\C^\times_{\text{loop}}}(\cQ)$ by
\cite{2016arXiv160403625B}. It should be the same, i.e.\ the first
action is given by the multiplication (left or right ?) through the
second homomorphism. Can we prove it ? \question

\subsection{Quiver description of the variety of triples}

To be written.

%%% Local Variables:
%%% mode: latex
%%% TeX-master: "shifted_K"
%%% End:
